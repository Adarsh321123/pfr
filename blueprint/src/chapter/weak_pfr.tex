\chapter{Weak PFR over the integers}

\begin{lemma}\label{torsion-free-doubling}\lean{torsion_free_doubling}
If $G$ is torsion-free and $X,Y$ are $G$-valued random variables then $d[X;2Y]\leq 5d[X;Y]$.
\end{lemma}
\begin{proof}
\uses{alt-submodularity,ruzsa-triangle,ruzsa-diff}
Let $Y_1,Y_2$ be independent copies of $Y$ (also independent of $X$). Since $G$ is torsion-free we know $X,Y_1-Y_2,X-2Y_1$ uniquely determine $X,Y_1,Y_2$ and so
\[\mathbb{H}(X,Y_1,Y_2,X-2Y_1)=\mathbb{H}(X,Y_1,Y_2)=\mathbb{H}(X)+2\mathbb{H}(Y).\]
Similarly
\[\mathbb{H}(X,X-2Y_1)=\mathbb{H}(X)+\mathbb{H}(2Y_1)=\mathbb{H}(X)+\mathbb{H}(Y).\]
Furthermore
\[\mathbb{H}(Y_1-Y_2,X-2Y_1)=\mathbb{H}(Y_1-Y_2,X-Y_1-Y_2)\leq \mathbb{H}(Y_1-Y_2)+\mathbb{H}(X-Y_1-Y_2).\]
By submodularity (Corollary \ref{alt-submodularity})
\[\mathbb{H}(X,Y_1,Y_2,X-2Y_1)+\mathbb{H}(X-2Y_1)\leq \mathbb{H}(X,X-2Y_1)+\mathbb{H}(Y_1-Y_2,X-2Y_1).\]
Combining these inequalities
\[\mathbb{H}(X-2Y_1)\leq \mathbb{H}(Y_1-Y_2)+\mathbb{H}(X-Y_1-Y_2)-\mathbb{H}(Y).\]
Similarly we have
\[\mathbb{H}(Y_1,Y_2,X-Y_1-Y_2)=\mathbb{H}(X)+2\mathbb{H}(Y),\]
\[\mathbb{H}(Y_1,X-Y_1-Y_2)=\mathbb{H}(Y)+\mathbb{H}(X-Y_2),\]
and
\[\mathbb{H}(Y_2,X-Y_1-Y_2)=\mathbb{H}(Y)+\mathbb{H}(X-Y_1)\]
and by submodularity (Corollary \ref{alt-submodularity}) again
\[\mathbb{H}(Y_1,Y_2,X-Y_1-Y_2)+ \mathbb{H}(X-Y_1-Y_2)\leq \mathbb{H}(Y_1,X-Y_1-Y_2)+\mathbb{H}(Y_2,X-Y_1-Y_2).\]
Combining these inequalities (and recalling the definition of Ruzsa distance) gives
\[\mathbb{H}(X-Y_1-Y_2)\leq \mathbb{H}(X-Y_1)+\mathbb{H}(X-Y_2)-\mathbb{H}(X)=2d[X;Y]+\mathbb{H}(Y).\]
It follows that
\[\mathbb{H}(X-2Y_1)\leq \mathbb{H}(Y_1-Y_2)+2d[X;Y]\]
and so (using $\mathbb{H}(2Y)=\mathbb{H}(Y)$)
\begin{align*}
d[X;2Y]
&=\mathbb{H}(X-2Y_1)-\mathbb{H}(X)/2-\mathbb{H}(2Y)/2\\
&\leq \mathbb{H}(Y_1-Y_2)+2d[X;Y]-\mathbb{H}(X)/2-\mathbb{H}(Y)/2\\
&= d[Y_1;Y_2]+\frac{\mathbb{H}(Y)-\mathbb{H}(X)}{2}+2d[X;Y].
\end{align*}
Finally note that by the triangle inequality (Lemma \ref{ruzsa-triangle}) we have
\[d[Y_1;Y_2]\leq d[Y_1;X]+d[X;Y_2]=2d[X;Y].\]
The result follows from $(\mathbb{H}(Y)-\mathbb{H}(X))/2\leq d[X;Y]$ (Lemma \ref{ruzsa-diff}).
\end{proof}

\begin{lemma}\label{torsion-dist-shrinking}\lean{torsion_dist_shrinking}
If $G$ is a torsion-free group and $X,Y$ are $G$-valued random variables and $\phi:G\to \mathbb{F}_2^d$ is a homomorphism then
\[\mathbb{H}(\phi(X))\leq 10d[X;Y].\]
\end{lemma}
\begin{proof}
\uses{fibring-ineq, torsion-free-doubling, dist-zero}
By Corollary \ref{fibring-ineq} and Lemma \ref{torsion-free-doubling} we have
\[d[\phi(X);\phi(2Y)]\leq d[X;2Y]\leq 5d[X;Y]\]
and $\phi(2Y)=2\phi(Y)\equiv 0$ so the left-hand side is equal to $d[\phi(X);0]=\mathbb{H}(\phi(X))/2$ (using Lemma \ref{dist-zero}).
\end{proof}

\begin{lemma}\label{app-ent-pfr}\lean{app_ent_PFR}\leanok
Let $G=\mathbb{F}_2^n$ and $X,Y$ be $G$-valued random variables such that
\[\mathbb{H}(X)+\mathbb{H}(Y)> 44d[X;Y].\]
There is a non-trivial subgroup $H\leq G$ such that
\[\log \lvert H\rvert <\mathbb{H}(X)+\mathbb{H}(Y)\] and
\[\mathbb{H}(\psi(X))+\mathbb{H}(\psi(Y))< \frac{\mathbb{H}(X)+\mathbb{H}(Y)}{2}\]
where $\psi:G\to G/H$ is the natural projection homomorphism.
\end{lemma}
\begin{proof}
\uses{entropy-pfr-improv, ruzsa-diff, dist-projection, ruzsa-nonneg}
By Theorem \ref{entropy-pfr-improv} there exists a subgroup $H$ such that $d[X;U_H]\leq \frac{11}{2} d[X;Y]$ and $d[Y;U_H]\leq \frac{11}{2} d[X;Y]$. Using Lemma \ref{dist-projection} we deduce that $\mathbb{H}(\psi(X))\leq 11 d[X;Y]$ and $\mathbb{H}(\psi(Y))\leq 11d[X;Y]$. The second claim follows adding these inequalities and using the assumption on $\mathbb{H}(X)+\mathbb{H}(Y)$.

Furthermore we have by Lemma \ref{ruzsa-diff}
\[\log \lvert H \rvert-\mathbb{H}(X)\leq 2d[X;U_H]\leq 11d[X;Y]\]
and similarly for $Y$ and thus
\[\log \lvert  H\rvert \leq \frac{\mathbb{H}(X)+\mathbb{H}(Y)}{2}+12d[X;Y]< \mathbb{H}(X)+\mathbb{H}(Y).\]
Finally note that if $H$ were trivial then $\psi(X)=X$ and $\psi(Y)=Y$ and hence $\mathbb{H}(X)+\mathbb{H}(Y)=0$, which contradicts Lemma \ref{ruzsa-nonneg}.
\end{proof}


\begin{lemma}\label{pfr-projection}\lean{PFR_projection}\leanok
If $G=\mathbb{F}_2^d$ and $X,Y$ are $G$-valued random variables then there is a subgroup $H\leq \mathbb{F}_2^d$ such that
\[\log \lvert H\rvert \leq 2(\mathbb{H}(X)+\mathbb{H}(Y))\]
and if $\psi:G \to G/H$ is the natural projection then
\[\mathbb{H}(\psi(X))+\mathbb{H}(\psi(Y))\leq 44 d[\psi(X);\psi(Y)].\]
\end{lemma}
\begin{proof}
\uses{app-ent-pfr}
Let $H\leq \mathbb{F}_2^d$ be a maximal subgroup such that
\[\mathbb{H}(\psi(X))+\mathbb{H}(\psi(Y))> 44d[\psi(X);\psi(Y)]\]
and
\[\log \lvert H\rvert \leq \left(2-2^{2-\lvert H\rvert}\right)(\mathbb{H}(X)+\mathbb{H}(Y))\]
and
\[\mathbb{H}(\psi(X))+\mathbb{H}(\psi(Y))\leq 2^{1-\lvert H\rvert}(\mathbb{H}(X)+\mathbb{H}(Y)).\]
Note that this exists since $H=\{0\}$ is an example of such a subgroup or we are done with this choice of $H$.

We know that $G/H$ has the shape $\mathbb{F}_2^{d'}$ for some $d'$ [exactly how to see this/set up the hypotheses of Lemma \ref{app-ent-pfr} to make this an easy deduction I'll leave to those who know Lean better] and so by Lemma \ref{app-ent-pfr} there exists some non-trivial subgroup $H'\leq G/H$ such that
\[\log \lvert H'\rvert < \mathbb{H}(\psi(X))+\mathbb{H}(\psi(Y))\]
and
\[2(\mathbb{H}(\psi'(X))+\mathbb{H}(\psi'(Y)))< \mathbb{H}(\psi(X))+\mathbb{H}(\psi(Y))\]
where $\psi':G/H\to (G/H)/H'$. By group isomorphism theorems we know that there exists some $H''$ with $H\leq H''\leq G$ such that $H'\cong H''/H$ and $\psi'(X)=\psi''(X)$ where $\psi'':G\to G/H''$ is the projection homomorphism.

Since $H'$ is non-trivial we know that $H$ is a proper subgroup of $H''$. On the other hand we know that
\[\log \lvert H''\rvert=\log \lvert H'\rvert+\log \lvert H\rvert< (2-2^{1-\lvert H\rvert})(\mathbb{H}(X)+\mathbb{H}(Y)).\]
and
\[\mathbb{H}(\psi''(X))+\mathbb{H}(\psi''(Y))< \frac{\mathbb{H}(\psi(X))+\mathbb{H}(\psi(Y))}{2}\leq 2^{1-\lvert H'\rvert}(\mathbb{H}(X)+\mathbb{H}(Y)).\]
Therefore (using the maximality of $H$) it must be the first condition that fails, whence
\[\mathbb{H}(\psi''(X))+\mathbb{H}(\psi''(Y))\leq 48d[\psi''(X);\psi''(Y)].\]
\end{proof}


\begin{lemma}\label{single-fibres}\lean{single_fibres}\leanok
Let $\phi:G\to H$ be a homomorphism and $A,B\subseteq G$ be finite subsets. If $x,y\in H$ then let $A_x=A\cap \phi^{-1}(x)$ and $B_y=B\cap \phi^{-1}(y)$. There exist $x,y\in H$ such that $A_x,B_y$ are both non-empty and
\[d[\phi(U_A);\phi(U_B)]\log \frac{\lvert A\rvert\lvert B\rvert}{\lvert A_x\rvert\lvert B_y\rvert}\leq (\mathbb{H}(\phi(U_A))+\mathbb{H}(\phi(U_B)))(d(U_A,U_B)-d(U_{A_x},U_{B_y})).\]
\end{lemma}
\begin{proof}
\uses{fibring-ident}
The random variables $(U_A\mid \phi(U_A)=x)$ and $(U_B\mid \phi(U_B)=y)$ are equal in distribution to $U_{A_x}$ and $U_{B_y}$ respectively (both are uniformly distributed over their respective fibres). It follows from Lemma \ref{fibring-ident} that
\begin{align*}
\sum_{x,y\in H}\frac{\lvert A_x\rvert\lvert B_y\rvert}{\lvert A\rvert\lvert B\rvert}d[U_{A_x};U_{B_y}]
&=d[U_A\mid \phi(U_A); U_B\mid \phi(U_B)]\\
&\leq d[U_A;U_B]-d[\phi(U_A);\phi(U_B)].
\end{align*}
Therefore with $M:=\mathbb{H}(\phi(U_A))+\mathbb{H}(\phi(U_B))$ we have
\[\sum_{x,y\in H}\frac{\lvert A_x\rvert\lvert B_y\rvert}{\lvert A\rvert\lvert B\rvert}Md[U_{A_x};U_{B_y}]+Md[\phi(U_A);\phi(U_B)]\leq Md[U_A;U_B].\]
Since
\[M=\sum_{x,y\in H}\frac{\lvert A_x\rvert\lvert B_y\rvert}{\lvert A\rvert\lvert B\rvert}\log \frac{\lvert A\rvert\lvert B\rvert}{\lvert A_x\rvert\lvert B_y\rvert}\]
we have
\[\sum_{x,y\in H} \frac{\lvert A_x\rvert\lvert B_y\rvert}{\lvert A\rvert\lvert B\rvert}\left(Md[U_{A_x};U_{B_y}]+d[\phi(U_A);\phi(U_B)]\log \frac{\lvert A_x\rvert\lvert B_y\rvert}{\lvert A\rvert\lvert B\rvert}\right)\leq  Md[U_A;U_B].\]
It follows that there exists some $x,y\in H$ such that $\lvert A_x\rvert,\lvert B_y\rvert\neq 0$ and
\[Md[U_{A_x};U_{B_y}]+d[\phi(U_A);\phi(U_B)]\log \frac{\lvert A_x\rvert\lvert B_y\rvert}{\lvert A\rvert\lvert B\rvert}\leq  Md[U_A;U_B].\]
\end{proof}


\begin{definition}\label{dimension-def}\lean{dimension}\leanok
If $A\subseteq \mathbb{Z}^{d}$ then by $\dim(A)$ we mean the dimension of the span of $A-A$ over the reals -- equivalently, the smallest $d'$ such that $A$ lies in a coset of a subgroup isomorphic to $\mathbb{Z}^{d'}$.
\end{definition}


\begin{theorem}\label{weak-pfr-asymm}\lean{weak_PFR_asymm}\leanok
If $A,B\subseteq \mathbb{Z}^d$ are finite non-empty sets then there exist non-empty $A'\subseteq A$ and $B'\subseteq B$ such that
\[\log\frac{\lvert A\rvert\lvert B\rvert}{\lvert A'\rvert\lvert B'\rvert}\leq 44d[U_A;U_B]\]
such that $\max(\dim A',\dim B')\leq \frac{40}{\log 2} d[U_A;U_B]$.
\end{theorem}
\begin{proof}
\uses{torsion-dist-shrinking, pfr-projection, single-fibres, dimension-def}
Without loss of generality we can assume that $A$ and $B$ are not both inside (possibly distinct) cosets of the same subgroup of $\mathbb{Z}^d$, or we just replace $\mathbb{Z}^d$ with that subgroup. We prove the result by induction on $\lvert A\rvert+\lvert B\rvert$.

Let $\phi:\mathbb{Z}^d\to \mathbb{F}_2^d$ be the natural mod-2 homomorphism. By Lemma \ref{torsion-dist-shrinking}
\[\max(\mathbb{H}(\phi(U_A)),\mathbb{H}(\phi(U_B)))\leq 10d[U_A;U_B].\]
We now apply Lemma \ref{pfr-projection}, obtaining some subgroup $H\leq \mathbb{F}_2^d$ such that
\[\log \lvert H\rvert \leq 40d[U_A;U_B]\]
and
\[\mathbb{H}(\tilde{\phi}(U_A))+\mathbb{H}(\tilde{\phi}(U_B))\leq 44 d[\tilde{\phi}(U_A);\tilde{\phi}(U_B)]\]
where $\tilde{\phi}:\mathbb{Z}^d\to \mathbb{F}_2^d/H$ is $\phi$ composed with the projection onto $\mathbb{F}_2^d/H$.


By Lemma \ref{single-fibres} there exist $x,y\in \mathbb{F}_2^d/H$ such that, with $A_x=A\cap \tilde{\phi}^{-1}(x)$ and similarly for $B_y$,
\[\log \frac{\lvert A\rvert\lvert B\rvert}{\lvert A_x\rvert\lvert B_y\rvert}\leq 44(d[U_A;U_B]-d[U_{A_x};U_{B_y}]).\]
Suppose first that $\lvert A_x\rvert+\lvert B_y\rvert=\lvert A\rvert+\lvert B\rvert$. This means that $\tilde{\phi}(A)=\{x\}$ and $\tilde{\phi}(B)=\{y\}$, and hence both $A$ and $B$ are in cosets of $\ker \tilde{\phi}$. Since by assumption $A,B$ are not in cosets of a proper subgroup of $\mathbb{Z}^d$ this means $\ker\tilde{\phi}=\mathbb{Z}^d$, and so (examining the definition of $\tilde{\phi}$) we must have $H=\mathbb{F}_2^d$. Then our bound on $\log\lvert H\rvert$ forces $d\leq \frac{40}{\log 2}d[U_A;U_B]$ and we are done with $A'=A$ and $B'=B$.

Otherwise,
\[\lvert A_x\rvert+\lvert B_y\rvert <\lvert A\rvert+\lvert B\rvert\]
and $\mathbb{H}(\tilde{\phi}(U_A))+\mathbb{H}(\tilde{\phi}(U_B))>0$, whence $d[\tilde{\phi}(U_A);\tilde{\phi}(U_B)]>0$. By induction we can find some $A'\subseteq A_x$ and $B'\subseteq B_y$ such that $\dim A',\dim B'\leq \frac{40}{\log 2} d[U_{A_x};U_{B_y}]\leq \frac{40}{\log 2}d[U_A;U_B]$ and

\[\log \frac{\lvert A_x\rvert\lvert B_y\rvert}{\lvert A'\rvert\lvert B'\rvert}\leq 44d[U_{A_x};U_{B_y}].\]
Adding these inequalities implies

\[\log\frac{\lvert A\rvert\lvert B\rvert}{\lvert A'\rvert\lvert B'\rvert}\leq 44d[U_A;U_B]\]
as required.
\end{proof}

\begin{theorem}\label{weak-pfr-symm}\lean{weak_PFR_symm}\leanok
If $A\subseteq \mathbb{Z}^d$ is a finite non-empty set with $d[U_A;U_A]\leq \log K$ then there exists a non-empty $A'\subseteq A$ such that
\[\lvert A'\rvert\geq K^{-44}\lvert A\rvert\]
and $\dim A'\leq 60\log K$.
\end{theorem}
\begin{proof}
\uses{weak-pfr-asymm,dimension-def}
Immediate from Theorem \ref{weak-pfr-asymm} and rearranging.
\end{proof}

\begin{theorem}\label{weak-pfr-int}\lean{weak_PFR_int}\leanok
Let $A\subseteq \mathbb{Z}^d$ and $\lvert A+A\rvert\leq K\lvert A\rvert$. There exists $A'\subseteq A$ such that $\lvert A'\rvert \geq K^{-44}\lvert A\rvert$ and $\dim A' \leq 60\log K$.
\end{theorem}
\begin{proof}
\uses{weak-pfr-symm,dimension-def}
As in the beginning of Theorem \ref{pfr} the doubling condition forces $d[U_A;U_A]\leq \log K$, and then we apply Theorem \ref{weak-pfr-symm}.
\end{proof}
