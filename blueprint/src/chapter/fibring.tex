\chapter{The Fibring lemma}

\begin{proposition}[General fibring identity]\label{fibring-ident}
  \lean{rdist_of_indep_eq_sum_fibre, rdist_le_sum_fibre}\leanok
  Let $\pi : H \to H'$ be a homomorphism additive groups, and let $Z_1,Z_2$ be $H$-valued random variables. Then we have
  \[
    d[Z_1; Z_2] \geq d[\pi(Z_1);\pi(Z_2)] + d[Z_1|\pi(Z_1); Z_2 |\pi(Z_2)].
  \]
  Moreover, if $Z_1,Z_2$ are taken to be independent, then the difference between the two sides is
$$I( Z_1 - Z_2 : (\pi(Z_1), \pi(Z_2))  |  \pi(Z_1 - Z_2) ).$$
\end{proposition}


\begin{proof}\uses{ruz-copy, independent-exist, submodularity,conditional-mutual-alt,chain-rule,relabeled-entropy, cond-dist-alt}\leanok
  Let $Z_1,Z_2$ be independent throughout (this is possible by Lemma \ref{ruz-copy} and Lemma \ref{independent-exist}).  By Lemma \ref{cond-dist-alt}, We have
  \begin{align*}
    & d[Z_1  |\pi(Z_1); Z_2 |\pi(Z_2)] \\
    & = \bbH[Z_1 - Z_2 | \pi(Z_1),\pi(Z_2)] - \tfrac{1}{2} \bbH[Z_1 | \pi(Z_1)] - \tfrac{1}{2} \bbH[Z_2 | \pi(Z_2)] \\
    & \leq  \bbH[Z_1 - Z_2 | \pi(Z_1+Z_2)]  - \tfrac{1}{2} \bbH[Z_1 | \pi(Z_1)] - \tfrac{1}{2}H[Z_2 | \pi(Z_2)] \\
    & = d[Z_1;Z_2] - d[\pi(Z_1);\pi(Z_2)].
  \end{align*}
  In the middle step, we used Lemma \ref{submodularity}, and in the last step we used the fact that
  \[\bbH[Z_1 - Z_2 |  \pi(Z_1-Z_2)] = \bbH[Z_1 - Z_2] - \bbH[\pi(Z_1-Z_2)]\]
  (thanks to Lemma \ref{chain-rule} and Lemma \ref{relabeled-entropy}) and that
  \[\bbH[Z_i| \pi(Z_i)] = \bbH[Z_i] - \bbH[\pi(Z_i)]\] (since $Z_i$ determines $\pi(Z_i)$).
  This gives the claimed inequality. The difference between the two sides is precisely
  \[\bbH[Z_1 - Z_2  | \pi(Z_1 - Z_2)] - \bbH[Z_1 - Z_2  | \pi(Z_1),\pi(Z_2)].\]
  To rewrite this in terms of (conditional) mutual information, we use the identity
  \[\bbH[A|B] - \bbH[A | B,C] = \bbI[A : C | B],\]
  (which follows Lemma \ref{conditional-mutual-alt})
  taking
  $A := Z_1 - Z_2$, $B := \pi(Z_1 - Z_2)$ and $C := (\pi(Z_1),\pi(Z_{2}))$, and noting that in this case $\bbH[A | B,C] = \bbH[A | C]$ since $C$ uniquely determines $B$ (this may require another helper lemma about entropy).
  This completes the proof.
\end{proof}


\begin{corollary}\label{fibring-ineq}
 \lean{rdist_of_hom_le} \leanok
If $\pi:G\to H$ is a homomorphism of additive groups and $X,Y$ are $G$-valued random variables then
\[d[X;Y]\geq d[\pi(X);\pi(Y)].\]
\end{corollary}

\begin{proof}
\uses{fibring-ident, ruzsa-nonneg}\leanok
By Proposition \ref{fibring-ident} and the nonnegativity of conditional Ruzsa distance (from Lemma \ref{ruzsa-nonneg}) we have
\[d[X;Y]\geq d[\pi(X);\pi(Y)]+d[X\mid \pi(X);Y\mid \pi(Y)].\]
The inequality follows from $d[X\mid \pi(X);Y\mid \pi(Y)]\geq 0$ (Lemma \ref{ruzsa-nonneg}).
\end{proof}


\begin{corollary}[Specific fibring identity]\label{cor-fibre}
  \lean{sum_of_rdist_eq}\leanok
  Let $Y_1,Y_2,Y_3$ and $Y_4$ be independent $G$-valued random variables.
  Then
\begin{align*}
  & d[Y_1+Y_3; Y_2+Y_4] + d[Y_1|Y_1+Y_3; Y_2|Y_2+Y_4] \\
  &\qquad + \bbI[Y_1+Y_2 : Y_2 + Y_4 | Y_1+Y_2+Y_3+Y_4] = d[Y_1; Y_2] + d[Y_3; Y_4].
\end{align*}
\end{corollary}

\begin{proof}
  \uses{fibring-ident, add_entropy, relabeled-entropy-cond}\leanok
  We apply Proposition \ref{fibring-ident} with $H := G \times G$, $H' := G$, $\pi$ the addition homomorphism $\pi(x,y) := x+y$, and with the random variables $Z_1 := (Y_1,Y_3)$ and $Z_2 := (Y_2,Y_4)$.  Then by independence (Lemma \ref{add-entropy})
\[
  d[Z_1; Z_2] = d[Y_1; Y_2] + d[Y_3; Y_4]
\]
while by definition
\[
  d[\pi(Z_1); \pi(Z_2)] = d[Y_1+Y_3; Y_2+Y_4].
\]
Furthermore,
\[
  d[Z_1|\pi(Z_1); Z_2|\pi(Z_2)] = d[Y_1|Y_1+Y_3;Y_2|Y_2+Y_4],
\]
since $Z_1=(Y_1,Y_3)$ and $Y_1$ are linked by an invertible affine transformation once $\pi(Z_1)=Y_1+Y_3$ is fixed, and similarly for $Z_2$ and $Y_2$.  (This has to do with Lemma \ref{relabeled-entropy-cond})
Finally, we have
\begin{align*}
  &\bbI[Z_1 + Z_2 : (\pi(Z_1),\pi(Z_2)) \,|\, \pi(Z_1) + \pi(Z_2)] \\
  &\ = \bbI[(Y_1+Y_2, Y_3+Y_4) : (Y_1+Y_3, Y_2+Y_4) \,|\, Y_1+Y_2+Y_3+Y_4] \\
  &\ = \bbI[Y_1+Y_2 : Y_2+Y_4 \,|\, Y_1+Y_2+Y_3+Y_4]
\end{align*}
where in the last line we used the fact that $(Y_1+Y_2, Y_1+Y_2+Y_3+Y_4)$ uniquely determine $Y_3+Y_4$ and similarly
$(Y_2+Y_4, Y_1+Y_2+Y_3+Y_4)$ uniquely determine $Y_1+Y_3$.  (This requires another helper lemma about entropy.)
\end{proof}
