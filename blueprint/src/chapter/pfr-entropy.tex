\chapter{Entropy version of PFR}

\begin{definition}\label{eta-def}
 \lean{eta}\leanok
  $\eta := 1/9$.
\end{definition}

Throughout this chapter,  $G = \F_2^n$, and $X^0_1, X^0_2$ are $G$-valued random variables.

\begin{definition}[$\tau$ functional]\label{tau-def}
  \uses{eta-def}
  \lean{tau}
If $X_1,X_2$ are two $G$-valued random variables, then
$$  \tau[X_1; X_2] \coloneqq \dist{X_1}{X_2} + \eta  \dist{X^0_1}{X_1} + \eta \dist{X^0_2}{X_2}.$$
\end{definition}

\begin{lemma}[$\tau$ depends only on distribution]\label{tau-copy}
  \uses{tau-def}
  \lean{tau-copy}  If $X'_1, X'_2$ are copies of $X_1,X_2$, then $\tau[X'_1;X'_2] = \tau[X_1;X_2]$.
\end{lemma}


\begin{proof}\uses{copy-ent} Immediate from Lemma \ref{copy-ent}.
\end{proof}

\begin{definition}[$\tau$-minimizer]\label{tau-min-def}
\uses{tau-def}
\lean{tau-minimizes}
A pair of $G$-valued random variables $X_1, X_2$ are said to be a $\tau$-minimizer if one has
  $$\tau[X_1;X_2] \leq \tau[X'_1;X'_2]
  $$
for all $G$-valued random variables $X'_1, X'_2$.
\end{definition}


\begin{proposition}[$\tau$ has minimum]\label{tau-min}\uses{tau-minimizes}
  \lean{tao-min-exists}
A pair $X_1, X_2$ of $\tau$-minimizers exist.
\end{proposition}

\begin{proof} By Lemma \ref{tau-copy}, $\tau$ only depends on the probability distributions of $X_1, X_2$. This ranges over a compact space, and $\tau$ is continuous.  So $\tau$ has a minimum.
\end{proof}

\subsection{Basic facts about minimizers}

In this section we assume that $X_1,X_2$ are $\tau$-minimizers. We also write $k := d[X_1;X_2]$.

\begin{lemma}[Distance lower bound]\label{distance-lower}
  \uses{tau-minimizes, ruz-dist-def}
  \lean{distance_ge_of_min}
  For any $G$-valued random variables $X'_1,X'_2$, one has
$$ d[X'_1;X'_2] \geq k - \eta (d[X^0_1;X'_1] - d[X^0_1;X_1] ) - - \eta (d[X^0_2;X'_2] - d[X^0_2;X_2] ).$$
\end{lemma}

\begin{proof}
  \uses{tau-def, tau-minimizes}
  Immediate from Definition \ref{tau-def} and Definition \ref{tau-minimizes}.
\end{proof}

\begin{lemma}[Conditional distance lower bound]\label{cond-distance-lower}
  \uses{tau-minimizes, cond-dist-def, ruz-dist-def}
  \lean{condDistance_ge_of_min}
  For any $G$-valued random variables $X'_1,X'_2$ and random variables $Z,W$, one has
$$ d[X'_1|Z;X'_2|W] \geq k - \eta (d[X^0_1;X'_1|Z] - d[X^0_1;X_1] ) - - \eta (d[X^0_2;X'_2|W] - d[X^0_2;X_2] ).$$
\end{lemma}

\begin{proof}\uses{distance-lower}  Apply Lemma \ref{distance-lower} to conditioned random variables and then average.
\end{proof}

\subsection{First estimate}

...

\subsection{Second estimate}

...

\subsection{Endgame}

...

\begin{theorem}[$\tau$-decrement]\label{de-prop}
  \uses{tau-def, ruz-dist-def}
  \lean{tau-strictly-decreases}
  Let $X_1, X_2$ be two $G$-valued random variables with $d[X_1;X_2] > 0$. Then there are $G$-valued random variables $X'_1, X'_2$ such that
$$\tau[X'_1;X'_2] < \tau[X_1;X_2].
$$
\end{theorem}

\begin{proof}  Sorry!  This is hard!
\end{proof}


\subsection{Conclusion}

\begin{theorem}[Entropy version of PFR]\label{entropy-pfr}\uses{ruz-dist-def}
  \lean{entropic_PFR_conjecture}
  Let $G = \F_2^n$, and suppose that $X^0_1, X^0_2$ are $G$-valued random variables.
  Then there is some subgroup $H \leq G$ such that
  \[
    d[X^0_1;U_H] + d[X^0_2;U_H] \le 11 d[X^0_1;X^0_2],
  \]
  where $U_H$ is uniformly distributed on $H$.denotes the uniform distribution on $H$.
  Furthermore, both $\dist{X^0_1}{U_H}$ and $\dist{X^0_2}{U_H}$ are at most $6 \dist{X^0_1}{X^0_2}$.
\end{theorem}

Probably need a definition for a uniform distribution on a non-empty subset $H$ of a set $S$.

\begin{proof} \uses{de-prop, tau-min, lem:100pc, ruzsa-triangle}  Let $X_1, X_2$ be the $\tau$-minimizer from Lemma \ref{tau-min}.  From Theorem \ref{de-prop}, $d[X_1;X_2]=0$.  From Lemma \ref{lem:100pc}, $d[X_1;U_H] = d[X_2; U_H] = 0$.  Also from $\tau$-minimization we have $\tau[X_1;X_2] \leq \tau[X^0_1;X^0_2]$.  Using this and the Ruzsa triangle inequality we can conclude.
\end{proof}

Note: a ``stretch goal'' for this project would be to obtain a `decidable` analogue of this result (see the remark at the end of Section 2 for some related discussion).
