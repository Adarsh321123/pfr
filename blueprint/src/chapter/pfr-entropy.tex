\chapter{Entropy version of PFR}

\begin{definition}\label{eta-def}
 \lean{η}\leanok
  $\eta := 1/9$.
\end{definition}

Throughout this chapter,  $G = \F_2^n$, and $X^0_1, X^0_2$ are $G$-valued random variables.

\begin{definition}[$\tau$ functional]\label{tau-def}
  \uses{eta-def, ruz-dist-def}
  \lean{tau}\leanok
If $X_1,X_2$ are two $G$-valued random variables, then
$$  \tau[X_1; X_2] := d[X_1; X_2] + \eta  d[X^0_1; X_1] + \eta d[X^0_2; X_2].$$
\end{definition}

\begin{lemma}[$\tau$ depends only on distribution]\label{tau-copy}\leanok
  \uses{tau-def}
  \lean{ProbabilityTheory.IdentDistrib.tau_eq}  If $X'_1, X'_2$ are copies of $X_1,X_2$, then $\tau[X'_1;X'_2] = \tau[X_1;X_2]$.
\end{lemma}


\begin{proof}\uses{copy-ent}\leanok Immediate from Lemma \ref{copy-ent}.
\end{proof}

\begin{definition}[$\tau$-minimizer]\label{tau-min-def}
\uses{tau-def}
\lean{tau_minimizes}\leanok
A pair of $G$-valued random variables $X_1, X_2$ are said to be a $\tau$-minimizer if one has
  $$\tau[X_1;X_2] \leq \tau[X'_1;X'_2]
  $$
for all $G$-valued random variables $X'_1, X'_2$.
\end{definition}


\begin{proposition}[$\tau$ has minimum]\label{tau-min}\uses{tau-min-def}
  \lean{tau_minimizer_exists}\leanok
A pair $X_1, X_2$ of $\tau$-minimizers exist.
\end{proposition}

\begin{proof}\uses{tau-copy}\leanok By Lemma \ref{tau-copy}, $\tau$ only depends on the probability distributions of $X_1, X_2$. This ranges over a compact space, and $\tau$ is continuous.  So $\tau$ has a minimum.
\end{proof}

\subsection{Basic facts about minimizers}

In this section we assume that $X_1,X_2$ are $\tau$-minimizers. We also write $k := d[X_1;X_2]$.

\begin{lemma}[Distance lower bound]\label{distance-lower}
  \uses{tau-minimizes}\leanok
  \lean{distance_ge_of_min}
  For any $G$-valued random variables $X'_1,X'_2$, one has
$$ d[X'_1;X'_2] \geq k - \eta (d[X^0_1;X'_1] - d[X^0_1;X_1] ) - \eta (d[X^0_2;X'_2] - d[X^0_2;X_2] ).$$
\end{lemma}

\begin{proof}
  \uses{tau-def, tau-min}\leanok
  Immediate from Definition \ref{tau-def} and Definition \ref{tau-min}.
\end{proof}

\begin{lemma}[Conditional distance lower bound]\label{cond-distance-lower}
  \uses{tau-minimizes, cond-dist-def}
  \lean{condDistance_ge_of_min}\leanok
  For any $G$-valued random variables $X'_1,X'_2$ and random variables $Z,W$, one has
$$ d[X'_1|Z;X'_2|W] \geq k - \eta (d[X^0_1;X'_1|Z] - d[X^0_1;X_1] ) - \eta (d[X^0_2;X'_2|W] - d[X^0_2;X_2] ).$$
\end{lemma}

\begin{proof}\uses{distance-lower}\leanok  Apply Lemma \ref{distance-lower} to conditioned random variables and then average.
\end{proof}

\subsection{First estimate}

We continue the assumptions from the preceding section.

Let $X_1, X_2, \tilde X_1, \tilde X_2$ be independent random variables, with $X_1,\tilde X_1$ copies of $X_1$ and $X_2,\tilde X_2$ copies of $X_2$.  (This is possible thanks to Lemma \ref{independent-exist}.)

We also define the quantity
$$ I_1 :=  I [ X_1+X_2 : \tilde X_1 + X_2 | X_1+X_2+\tilde X_1+\tilde X_2 ].$$

\begin{lemma}[Fibring identity for first estimate]\label{first-fibre}
  \lean{rdist_add_rdist_add_condMutual_eq}\leanok
  We have
  \begin{align*}
    &   d[X_1+\tilde X_2;X_2+\tilde X_1] + d[X_1|X_1+\tilde X_2; X_2|X_2+\tilde X_1] \\
    &\quad + \bbI[ X_1+ X_2 : \tilde X_1 + X_2 \,|\, X_1 + X_2 + \tilde X_1 + \tilde X_2 ] = 2k.
  \end{align*}
\end{lemma}

\begin{proof}\uses{cor-fibre} \leanok  Immediate from Corollary \ref{cor-fibre}.
\end{proof}

\begin{lemma}[Lower bound on distances]\label{first-dist-sum}
  \lean{rdist_of_sums_ge}\leanok
  We have
  \begin{align*}
    d[X_1+\tilde X_2; X_2+\tilde X_1] \geq k &- \eta (d[X^0_1; X_1+\tilde X_2] - d[X^0_1; X_1]) \\& \qquad- \eta (d[X^0_2; X_2+\tilde X_1] - d[X^0_2; X_2])
  \end{align*}
\end{lemma}


\begin{proof}\uses{distance-lower}\leanok Immediate from Lemma \ref{distance-lower}.
\end{proof}

\begin{lemma}[Lower bound on conditional distances]\label{first-cond}
  \lean{cond_rdist_of_sums_ge}\leanok
  We have
  \begin{align*}
    & d[X_1|X_1+\tilde X_2; X_2|X_2+\tilde X_1]  \\  & \qquad\quad \geq k - \eta (d[X^0_1; X_1 | X_1 + \tilde X_2] - d[X^0_1; X_1]) \\
    & \qquad\qquad\qquad\qquad  - \eta(d[X^0_2; X_2 | X_2 + \tilde X_1] - d[X^0_2; X_2]).
  \end{align*}
\end{lemma}

\begin{proof}\uses{cond-distance-lower}\leanok Immediate from Lemma \ref{cond-distance-lower}.
\end{proof}

\begin{lemma}[Upper bound on distance differences]\label{first-upper}\leanok
  \lean{diff_rdist_le_1, diff_rdist_le_2, diff_rdist_le_3, diff_rdist_le_4}
  We have
  \begin{align*}
    d[X^0_1; X_1+\tilde X_2] - d[X^0_1; X_1] &\leq \tfrac{1}{2} k + \tfrac{1}{4} \bbH[X_2] - \tfrac{1}{4} \bbH[X_1]\\
    d[X^0_2;X_2+\tilde X_1] - d[X^0_2; X_2] &\leq \tfrac{1}{2} k + \tfrac{1}{4} \bbH[X_1] - \tfrac{1}{4} \bbH[X_2], \\
    d[X_1^0;X_1|X_1+\tilde X_2] - d[X_1^0;X_1] &\leq \tfrac{1}{2} k + \tfrac{1}{4} \bbH[X_1] - \tfrac{1}{4} \bbH[X_2]  \\
    d[X_2^0; X_2|X_2+\tilde X_1] - d[X_2^0; X_2] &\leq \tfrac{1}{2}k + \tfrac{1}{4} \bbH[X_2] - \tfrac{1}{4} \bbH[X_1].
  \end{align*}
\end{lemma}

\begin{proof}\uses{first-useful} \leanok  Immediate from Lemma \ref{first-useful} (and recalling that $k$ is defined to be $d[X_1;X_2]$).
\end{proof}

\begin{lemma}[First estimate]\label{first-estimate}
  \lean{first_estimate}\leanok We have $I_1 \leq 2 \eta k$.
\end{lemma}

\begin{proof}\uses{first-fibre, first-dist-sum, first-cond, first-upper}\leanok  Take a suitable linear combiation of Lemma \ref{first-fibre}, Lemma \ref{first-dist-sum}, Lemma \ref{first-cond}, and Lemma \ref{first-upper}.
\end{proof}

One can also extract the following useful inequality from the proof of the above lemma.

\begin{lemma}[Entropy bound on quadruple sum]\label{foursum-bound}
  \lean{ent_ofsum_le}\leanok
  With the same notation, we have
  \begin{equation}
    \label{HS-bound}
    \bbH[X_1+X_2+\tilde X_1+\tilde X_2] \le \tfrac{1}{2} \bbH[X_1]+\tfrac{1}{2} \bbH[X_2] + (2 + \eta) k - I_1.
  \end{equation}
\end{lemma}

\begin{proof}\uses{first-cond, first-fibre, first-upper, ruz-indep}\leanok
  Subtracting Lemma \ref{first-cond} from Lemma \ref{first-fibre}, and combining the resulting inequality with Lemma \ref{first-upper} gives the bound
\[
  d[X_1+\tilde X_2;X_2+\tilde X_1] \le (1 + \eta) k - I_1,
\]
and the claim follows from Lemma \ref{ruz-indep} and the definition of $k$.
\end{proof}


\subsection{Second estimate}

We continue the assumptions from the preceding section.
We introduce the quantity
$$ I_2 := \bbI[ X_1+X_2 : X_1 + \tilde X_1 | X_1+X_2+\tilde X_1+\tilde X_2 ].$$

\begin{lemma}[Distance between sums]\label{dist-sums}
  \lean{rdist_of_sums_ge'}\leanok
We have
$$ d[X_1+\tilde X_1; X_2+\tilde X_2] \geq k - \frac{\eta}{2} ( d[X_1; X_1] + d[X_2;X_2] ).$$
\end{lemma}

\begin{proof}\uses{distance-lower, first-useful}\leanok From Lemma \ref{distance-lower} one has
\begin{align*}
    d[X_1+\tilde X_1; X_2+\tilde X_2] \geq k &- \eta(d[X^0_1;X_1] - d[X^0_1;X_1+\tilde X_1]) \\
    &- \eta(d[X^0_2;X_2] - d[X^0_2;X_2+\tilde X_2]).
\end{align*}
Now Lemma \ref{first-useful} gives
$$
d[X^0_1;X_1+\tilde X_1] - d[X^0_1;X_1] \leq \tfrac{1}{2} d[X_1;X_1]$$
and
$$
  d[X^0_2;X_2+\tilde X_2] - d[X^0_2;X_2] \leq \tfrac{1}{2} d[X_2;X_2],
$$
and the claim follows.
\end{proof}

\begin{lemma}[Second estimate]\label{second-estimate}
  \lean{second_estimate}\leanok
We have
$$ I_2 \leq 2 \eta k + \frac{2 \eta (2 \eta k - I_1)}{1 - \eta}.$$
\end{lemma}

\begin{proof}
  \uses{cor-fibre,distance-lower,first-useful,ruz-indep,dist-sums,foursum-bound}\leanok
  We apply Corollary \ref{cor-fibre}, but now with the choice
  \[
    (Y_1,Y_2,Y_3,Y_4) := (X_2, X_1, \tilde X_2, \tilde X_1).
  \]
  Now Corollary \ref{cor-fibre} can be rewritten as
  \begin{align*}
    &d[X_1+\tilde X_1;X_2+\tilde X_2] + d[X_1|X_1+\tilde X_1; X_2|X_2+\tilde X_2] \\
    &\quad + \bbI[ X_1+X_2 : X_1 + \tilde X_1 \,|\, X_1+X_2+\tilde X_1+\tilde X_2 ] = 2k,
  \end{align*}
  recalling once again that $k := d[X_1;X_2]$.  From Lemma \ref{cond-distance-lower} one has
  \begin{align*}
    d[X_1|X_1+\tilde X_1; X_2|X_2+\tilde X_2]   \geq k &- \eta (d[X^0_1;X_1] - d[X^0_1;X_1|X_1+\tilde X_1]) \\& - \eta (d[X^0_2;X_2] - d[X^0_2;X_2|X_2+\tilde X_2]) .
  \end{align*}
  while from Lemma \ref{first-useful} we have
  \[
    d[X^0_1;X_1|X_1+\tilde X_1] -  d[X^0_1;X_1] \leq  \tfrac{1}{2} d[X_1;X_1],
  \]
  and
  \[
    d[X^0_2;X_2|X_2+\tilde X_2] -  d[X^0_2;X_2] \leq \tfrac{1}{2} d[X_1;X_2].
    \]
  Combining all these inequalities with Lemma \ref{dist-sums}, we have
  \begin{equation}\label{combined}
  \bbI[ X_1+X_2 : X_1 + \tilde X_1 | X_1+X_2+\tilde X_1+\tilde X_2 ] \leq \eta ( d[X_1; X_1] + d[X_2; X_2] ).
  \end{equation}
  We may also use Lemma \ref{ruz-indep} to expand
  \begin{align*}
   & d[X_1+\tilde X_1;X_2+\tilde X_2] \\ &= \bbH[ X_1+\tilde X_1 + X_2 + \tilde X_2 ]  - \tfrac{1}{2} \bbH[X_1+\tilde X_1] - \tfrac{1}{2} \bbH[X_2+\tilde X_2] \\
    &= \bbH[ X_1+\tilde X_1 + X_2 + \tilde X_2]  - \tfrac{1}{2} \bbH[X_1] - \tfrac{1}{2} \bbH[X_2]  \\ & \qquad\qquad\qquad   - \tfrac{1}{2} \left( d[X_1;X_1] + d[X_2; X_2] \right),
  \end{align*}
  and hence by Lemma \ref{foursum-bound}
  \[
    d[X_1+\tilde X_1; X_2+\tilde X_2] \leq (2+\eta) k - \tfrac{1}{2} \left( d[X_1;X_1] + d[X_2;X_2] \right) - I_1.
  \]
  Combining this bound with Lemma \ref{dist-sums} we obtain
  \[ d[X_1;X_1] + d[X_2;X_2] \leq 2 k + \frac{2(2 \eta k - I_1)}{1-\eta}. \]
  Therefore by~\eqref{combined} we have the desired bound.
\end{proof}


\subsection{Endgame}

Let $X_1,X_2,\tilde X_1,\tilde X_2$ be as before, and introduce the random variables
\[ U := X_1 + X_2, \qquad V := \tilde X_1 + X_2, \qquad W := X_1 + \tilde X_1\] and
\[   S := X_1 + X_2 + \tilde X_1 + \tilde X_2.\]

\begin{lemma}[Symmetry identity]\label{symm-lemma}
  \lean{I₃_eq}\leanok
We have
$$ I(U:W | S) = I(V:W | S).$$
\end{lemma}

\begin{proof}\uses{copy-ent, conditional-mutual-alt, chain-rule}
This should follow from Lemma \ref{copy-ent}, Lemma \ref{conditional-mutual-alt}, and Lemma \ref{chain-rule}.
\end{proof}

\begin{lemma}[Bound on conditional mutual informations]\label{uvw-s}
\uses{conditional-information-def}
\lean{I₃_eq,sum_condMutual_le}\leanok
We have
$$
I(U : V \, | \, S) + I(V : W \, | \,S) + I(W : U \, | \, S) \leq 6 \eta k - \frac{1 - 5 \eta}{1-\eta} (2 \eta k - I_1).
$$
\end{lemma}

\begin{proof}
  \uses{second-estimate,symm-lemma}\leanok
From the definitions of $I_1,I_2$ and Lemma \ref{symm-lemma}, we see that
\[
  I_1 = I(U : V \, | \, S), \qquad I_2 = I(W : U \, | \, S), \qquad I_2 = I(V : W \, | \,S).
\]
Applying Lemma \ref{first-estimate} and Lemma \ref{second-estimate} we have the inequalities
\[   I_2 \leq 2 \eta k + \frac{2\eta(2 \eta k - I_1)}{1-\eta} .
\]
We conclude that
$$
   I_1 + I_2 + I_2 \leq I_1+4\eta k+ \frac{4\eta(2 \eta k - I_1)}{1-\eta}
$$
and the claim follows from some calculation.
\end{proof}

\begin{lemma}[Bound on distance increments]\label{total-dist}\leanok
  \lean{sum_dist_diff_le}
   We have
\begin{align*}
  \sum_{i=1}^2 \sum_{A\in\{U,V,W\}} \big(d[X^0_i;A|S] & - d[X^0_i;X_i]\big) \\
    &\leq (6 - 3\eta) k + 3(2 \eta k - I_1).
\end{align*}
\end{lemma}

\begin{proof}
  \uses{second-useful, foursum-bound}
  By Lemma \ref{second-useful} (taking $X = X_1^0$, $Y = X_1$, $Z = X_2$ and $Z' = \tilde X_1 + \tilde X_2$, so that $Y + Z = U$ and $Y + Z + Z' = S$) we have, noting that $\bbH[Y+Z] = \bbH[Z']$,
  \[
    d[X^0_1;U|S] - d[X^0_1;X_1] \leq \tfrac{1}{2} (\bbH[S] -  \bbH[X_1]).
  \]
  Further applications of Lemma \ref{second-useful} give
  \begin{align*}
  d[X^0_2;U|S] - d[X^0_2; X_2] &\leq \tfrac{1}{2} (\bbH[S] -  \bbH[X_2]) \\
  d[X^0_1;V|S] - d[X^0_1;X_1] &\leq \tfrac{1}{2} (\bbH[S] -  \bbH[X_1])\\
  d[X^0_2;V|S] - d[X^0_2;X_2] &\leq \tfrac{1}{2} (\bbH[S] -  \bbH[X_2])
  \end{align*}
  and
  \[ d[X^0_1;W|S] - d[X^0_1;X_1] \leq \tfrac{1}{2} (\bbH[S] + \bbH[W] - \bbH[X_1] - \bbH[W']),\] where $W' := X_2 + \tilde X_2$.
  To treat $d[X^0_2;W|S]$, first note that this equals $d[X^0_2;W'|S]$, since for a fixed choice $s$ of $S$ we have $W' = W + s$ (here we need some helper lemma about Ruzsa distance). Now we may apply Lemma \ref{second-useful} to obtain
  \[ d[X^0_2;W'|S] - d[X^0_2;X_2] \leq \tfrac{1}{2} (\bbH[S] + \bbH[W'] - \bbH[X_2] - \bbH[W]).\]
  Summing these six estimates and using Lemma \ref{foursum-bound}, we conclude that
  \begin{align*}
    \sum_{i=1}^2 \sum_{A\in\{U,V,W\}} \big(d[X^0_i;A|S] & - d[X^0_i;X_i]\big) \\
      &\leq 3 \bbH[S] - \tfrac{3}{2} \bbH[X_1] - \tfrac{3}{2} \bbH[X_2]\\
      &\leq (6 - 3\eta) k + 3(2 \eta k - I_1)
  \end{align*}
  as required.
\end{proof}

\begin{lemma}[Key identity]\label{key-ident}
\lean{sum_uvw_eq_zero}\leanok
We have $U+V+W=0$.
\end{lemma}

\begin{proof} \leanok Obvious because we are in characteristic two.
\end{proof}

For the next two lemmas, let $(T_1,T_2,T_3)$ be a $G^3$-valued random variable such that $T_1+T_2+T_3=0$ holds identically. Set
\begin{equation}\label{delta-t1t2t3-def}
  \delta := \sum_{1 \leq i < j \leq 3} \bbI[T_i;T_j].
\end{equation}

\begin{lemma}[Constructing good variables, I]\label{construct-good-prelim}
  \lean{construct_good_prelim}\leanok
  One has
  \begin{align*}  k \leq
    \delta + \eta (& d[X^0_1;T_1]-d[X^0_1;X_1])
      + \eta (d[X^0_2;T_2]-d[X^0_2;X_2]) \\ & + \tfrac12 \eta \bbI[T_1:T_3] + \tfrac12 \eta \bbI[T_2:T_3].
  \end{align*}
\end{lemma}

(Note: in the paper, this lemma was phrased in a more intuitive formulation that is basically the contrapositive of the one here. Similarly for the next two lemmas.)

\begin{proof}\uses{lem-bsg, cond-dist-fact,distance_lower}\leanok
  We apply Lemma \ref{lem-bsg} with $(A,B) = (T_1, T_2)$ there.
  Since $T_1 + T_2 = T_3$, the conclusion is that
  \begin{align} \nonumber \sum_{t_3} \bbP[T_3 = t_3] & d[(T_1 | T_3 = t_3); (T_2 | T_3 = t_3)] \\ & \leq 3 \bbI[T_1 : T_2] + 2 \bbH[T_3] - \bbH[T_1] - \bbH[T_2].\label{bsg-t1t2}
  \end{align}
  The right-hand side in~\eqref{bsg-t1t2} can be rearranged as
  \begin{align*} & 2( \bbH[T_1] + \bbH[T_2] + \bbH[T_3]) - 3 \bbH[T_1,T_2] \\ & = 2(\bbH[T_1] + \bbH[T_2] + \bbH[T_3]) - \bbH[T_1,T_2] - \bbH[T_2,T_3] - \bbH[T_1, T_3] = \delta,\end{align*}
  using the fact (from Lemma \ref{relabeled-entropy}) that all three terms $\bbH[T_i,T_j]$ are equal to $\bbH[T_1,T_2,T_3]$ and hence to each other.
  We also have
  \begin{align*}
  &  \sum_{t_3} P[T_3 = t_3] \bigl(d[X^0_1; (T_1 | T_3=t_3)] - d[X^0_1;X_1]\bigr) \\
  &\quad = d[X^0_1; T_1 | T_3] - d[X^0_1;X_1] \leq d[X^0_1;T_1] - d[X^0_1;X_1] + \tfrac{1}{2} \bbI[T_1 : T_3]
  \end{align*}
  by Lemma \ref{cond-dist-fact}, and similarly
  \begin{align*}
  &  \sum_{t_3} \bbP[T_3 = t_3] (d[X^0_2;(T_2 | T_3=t_3)] - d[X^0_2; X_2]) \\
  &\quad\quad\quad\quad\quad\quad \leq d[X^0_2;T_2] - d[X^0_2;X_2] + \tfrac{1}{2} \bbI[T_2 : T_3].
  \end{align*}
  Putting the above observations together, we have
  \begin{align*}
   \sum_{t_3} \bbP[T_3=t_3] \psi[(T_1 | T_3=t_3); (T_2 | T_3=t_3)] \leq \delta + \eta (d[X^0_1;T_1]-d[X^0_1;X_1]) \\
     + \eta (d[X^0_2;T_2]-d[X^0_2;X_2]) + \tfrac12 \eta \bbI[T_1:T_3] + \tfrac12 \eta \bbI[T_2:T_3]
   \end{align*}
where we introduce the notation
\[ \psi[Y_1; Y_2] := d[Y_1;Y_2] +  \eta (d[X_1^0;Y_1] - d[X_1^0;X_1]) + \eta(d[X_2^0;Y_2] - d[X_2^0;X_2]).\]
On the other hand, from Lemma \ref{distance-lower} we have $k \leq \psi[Y_1;Y_2]$, and the claim follows.
  \end{proof}


\begin{lemma}[Constructing good variables, II]\label{construct-good}
  \lean{construct_good}\leanok
One has
\begin{align*}  k & \leq  \delta + \frac{\eta}{3} \biggl( \delta + \sum_{i=1}^2 \sum_{j = 1}^3 (d[X^0_i;T_j] - d[X^0_i; X_i]) \biggr).
  \end{align*}
\end{lemma}

\begin{proof}
\uses{construct-good-prelim}\leanok
Average Lemma \ref{construct-good-prelim} over all six permutations of $T_1,T_2,T_3$.
\end{proof}




\begin{theorem}[$\tau$-decrement]\label{de-prop}
  \lean{tau_strictly_decreases}\leanok
  Let $X_1, X_2$ be tau-minimizers.  Then $d[X_1;X_2] = 0$.
\end{theorem}

\begin{proof}
  \uses{construct-good, key-ident, uvw-s, total-dist, first-estimate, eta-def}\leanok
  Set $k := d[X_1;X_2]$.  Applying Lemma \ref{construct-good} with any random variables $(T_1,T_2,T_3)$ such that $T_1+T_2+T_3=0$ holds identically, we deduce that
\[
  k \leq \delta + \frac{\eta}{3} \biggl( \delta + \sum_{i=1}^2 \sum_{j = 1}^3 (d[X^0_1;T_j] - d[X^0_i;X_i]) \biggr).
\]
Note that $\delta$ is still defined by~\eqref{delta-t1t2t3-def} and thus depends on $T_1,T_2,T_3$.
In particular we may apply this for
\[
  T_1 = (U | S = s),\qquad T_2 = (V | S = s), \qquad T_3 = (W | S = s)
\]
for $s$ in the range of $S$ (which is a valid choice by Lemma \ref{key-ident}) and then average over $s$ with weights $p_S(s)$, to obtain
\[ k \leq \tilde \delta + \frac{\eta}{3} \biggl( \tilde \delta + \sum_{i=1}^2 \sum_{A\in\{U,V,W\}} \bigl(  d[X^0_i;A|S] - d[X^0_i;X_i]\bigr) \biggr),\]
where
\[
  \tilde \delta :=  \bbI[U : V | S] + \bbI[V : W | S] + \bbI[W : U | S].
\]
Putting this together with Lemma \ref{uvw-s} and Lemma \ref{total-dist}, we conclude that
\begin{align*}
  k &\leq \Bigl(1+\frac{\eta}{3}\Bigr)\Bigl(6\eta k-\frac{1-5\eta}{1-\eta}(2\eta k-I_1)\Bigr)+\frac{\eta}{3}\Bigl((6-3\eta)k+3(2\eta k-I_1)\Bigr)\\
  &= (8\eta + \eta^2) k - \biggl( \frac{1 - 5 \eta}{1-\eta}\Bigl(1 + \frac{\eta}{3}\Bigr) -  \eta \biggr)(2 \eta k - I_1)\\
  &\leq (8 \eta + \eta^2) k
 \end{align*}
since the quantity $2 \eta k - I_1$ is non-negative (by Lemma \ref{first-estimate}), and its coefficient in the above expression is non-positive provided that $\eta(2\eta + 17) \le 3$, which is certainly the case with Definition \ref{eta-def}.
Moreover, from Definition \ref{eta-def} we have $8 \eta + \eta^2 < 1$. It follows that $k=0$, as desired.
\end{proof}


\subsection{Conclusion}

\begin{theorem}[Entropy version of PFR]\label{entropy-pfr}
  \lean{entropic_PFR_conjecture}\leanok
  Let $G = \F_2^n$, and suppose that $X^0_1, X^0_2$ are $G$-valued random variables.
  Then there is some subgroup $H \leq G$ such that
  \[
    d[X^0_1;U_H] + d[X^0_2;U_H] \le 11 d[X^0_1;X^0_2],
  \]
  where $U_H$ is uniformly distributed on $H$.denotes the uniform distribution on $H$.
  Furthermore, both $d[X^0_1;U_H]$ and $d[X^0_2;U_H]$ are at most $6 d[X^0_1;X^0_2]$.
\end{theorem}

\begin{proof} \uses{de-prop, tau-min, lem:100pc, ruzsa-triangle} \leanok  Let $X_1, X_2$ be the $\tau$-minimizer from Lemma \ref{tau-min}.  From Theorem \ref{de-prop}, $d[X_1;X_2]=0$.  From Corollary \ref{lem:100pc}, $d[X_1;U_H] = d[X_2; U_H] = 0$.  Also from $\tau$-minimization we have $\tau[X_1;X_2] \leq \tau[X^0_1;X^0_2]$.  Using this and the Ruzsa triangle inequality we can conclude.
\end{proof}

Note: a ``stretch goal'' for this project would be to obtain a `decidable` analogue of this result (see the remark at the end of Section 2 for some related discussion).
